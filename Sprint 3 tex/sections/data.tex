An essential data source for this research is the North Carolina Department of Public Instruction (NC DPI). 
It is the organization in charge of executing educational legislation \parencite[][]{NCDPI}. NC DPI governs public, charter, and educational institutions for students with hearing and vision impairments. 
The curriculum for North Carolina (NC) is developed by the NC DPI, as it administrates accountability, finance, and administrative work throughout NC schools. 
Licensing for NC teachers is also the responsibility of the NC DPI. It oversees the data collection, in efforts to aggregate accurate school records and accountability information in an organized, accessible format \parencite[][]{NCDPI}.

The North Carolina School Report Cards is a tool utilized by the NC DPI to compile demographics, scores, and other statistics on NC schools \parencite{NCReportCards}.
Academic performance and enrollment is measured in a variety of ways, from test scores to Advanced Placement (AP) classes. 
The information ranges in years, and some features have been retired in lieu of more accurate measurements, so this data set is a bit eclectic, but comprehensive \parencite{NCReportCards}. 

Another dataset handled by the NC DPI is the North Carolina Public Schools Statistical Profile \parencite{NCStats}.
It was established in 1975 to provide open, general statistics on NC Public Schools at the state, school district, school, and charter school level \parencite{NCStats}. 
An issue that arose with this dataset was the lack of school-level data on all features. For example, data exists at the district and school level for high school graduate post secondary intentions, but not for school personnel \parencite{NCStats}. 
When following this data to the source, Common Core Data managed by the National Center for Education Statistics, it was discovered that North Carolina either did not or could not provide this granular level of data for the Common Core survey \parencite{CCD}.
Future plans to deal with this problem are addressed later in this paper. 

Charlotte's Quality of Life Explorer inspects socioeconomic and structural conditions in Mecklenburg County \parencite{quality}.
Information regarding these topics are available as interactive maps, tables, and downloadable reports by neighborhood \parencite{quality}.
Reports can be generated by filtering data geographically, allowing for unique geospatial analysis, such as the `crescent and wedge' or the light rail corridor \parencite{quality}.
This source of data was used to find the 2019 median household income by zip code in Mecklenburg County. 
This is the current placeholder value for economic data per school, and will be used to separate schools into rough economic clusters \parencite{quality}. 
\\
\begin{threeparttable}
    \caption{Codebook} %% the caption, required by APA7 and it looks nice :D
\label{tab:codebook} %% use this to refer to the table within the paper (I think a link will open up)
    \begin{tabular}{ p{0.17\linewidth} p{0.115\linewidth} p{0.13\linewidth} p{0.49\linewidth}}     %% 5 columns, all centered. the tabular environment begins the actual table.    
    \toprule %% A line across the top
    Variable                        & \multicolumn{3}{c}{Information} \\ %% The titles of the columns. 

                       \cmidrule(r){2-4} %% a line going across columns 2 - end
                                    &   Years    &    Type                 &  Description \\ 
\midrule
                                    &               &  Report Card        &                                    \\ 
AP\_part\_pct  & 2014-2020 &  Continuous                   &   Percent of students enrolled in AP classes\\
AP\_pass\_pct & 2014-2020 &  Continuous                   &   Percent of AP exams with a score of 3 or more     \\
enroll\_\textbf{subgroup}\tabfnm{a}& 2011-2019   &  Continuous                  &   Percent of students enrolling in college           \\
 CTE\_enroll\_pct& 2018-2020  &  Continuous                  &  Percentage of students enrolled in a CTE program     \\
CTE\_cred\_pct  & 2018-2020  &  Continuous                 &  Ratio of CTE credentials earned over students enrolled in programs          \\
\midrule
                                    &               &  NC Stats         &                     \\ 
 total\_counselors & 2015-2020  &  Discrete                  &  Guidance counselors employed district wide  \\
 int\_\textbf{intention}\tabfnm{c} & 2015-2020 &  Continuous                   &  Percent of students by post-secondary intention           \\
\midrule
&               & ISC      & \\ 
Dual Enroll                     &  ---   &     Continuous         &   Percent of students dual-enrolled\tabfnm{b}            \\
\midrule
\end{tabular}
\begin{tablenotes}[para,flushleft]
    {\small
        \textit{Note.} 

        \tabfnt{a}See table \textbf[REFERENCE TABLE OF SUBGROUPS HERE] for all subgroups.\\
        \tabfnt{b}Low-SES students, if possible.\\
        \tabfnt{c}See table \textbf[REFERENCE TABLE OF SUBGROUPS HERE] for all graduate intentions.
     }
\end{tablenotes}
\end{threeparttable}

\vspace{2cm}
\begin{threeparttable}
    \renewcommand\thetable{2}
    \caption{Summary Statistics} %% the caption, required by APA7 and it looks nice :D
\label{tab:summarystats1} %% use this to refer to the table within the paper (I think a link will open up)
    \begin{tabular}{ p{0.26\linewidth} p{0.1\linewidth} p{0.1\linewidth} p{0.1\linewidth} p{0.1\linewidth} p{0.1\linewidth} p{0.1\linewidth}}     %% 5 columns, all centered. the tabular environment begins the actual table.    
    \toprule %% A line across the top
    Variable                        & \multicolumn{6}{c}{Statistics} \\ %% The titles of the columns. 

                       \cmidrule(r){2-7} %% a line going across columns 2 - end
                                    &   Non-Missing Count   &   Mean & Std & Min & Max & Missing  \\ 
\midrule 

 AP\_pass\_pct  &  195  &  0.406  & 0.23    & 0.05 & 0.87  & 42.14 \%  \\ 
 AP\_part\_pct  &  195  &  0.219  & 0.125   & 0    & 0.72  & 42.14 \%  \\ 
 int\_commcoll  &  337  &  0.35   & 0.162   & 0    & 1     & 0 \%  \\ 
 int\_pubsr     &  337  &  0.398  & 0.181   & 0    & 1     & 0 \%  \\ 
 int\_trdbusnrs &  337  &  0.016  & 0.02    & 0    & 0.139 & 0 \%  \\
 CTE\_cred\_pct &  24   &  0.143  & 0.14    & 0    & 0.54  & 92.878 \%\\
 CTE\_enroll\_pct& 78 & 0.661 & 0.148 & 0.221 & 0.98 & 76.86 \% \\
 enroll\_Disadvantaged & 209 & 0.378 & 0.219 & 0 & 0.919 & 37.982 \% \\
\midrule
\end{tabular}

\end{threeparttable}

Hypothesis 1 and 2 have both run into roadblocks in terms of operationalizing variables. In terms of operationally defined variables, the number of school counselors per high school is unavailable. 
Instead, only the district level aggregate data exists, broken down by primary and secondary schools. 
Although an increase is shown in total counselors every year, the economic disparities between Charlotte schools is not addressed with this operationalization. 
Social capital variables are the number of counselors and median household income. 
These are relatively poor proxies for social capital, considering we do not have the school level data on guidance counselors and median household income doubles as our only economic status indicator. 

Accountability laws and data collection standards are prone to change, which makes the data available change. 
CTE course enrollment and credentials earned by students was only publicly available in 2017 and on. 
This makes an analysis of Career and College Readiness before and after the formation of LOO difficult. 
At the same time, data that does have expanded years may be available in a retired dataset, but with different collection standards. 
For example, students participating in the College and Career Promise Program from CPCC is available from 2017 and on, however before this the data is only available for students enrolled in any postsecondary classes. 
Thus far, the hypotheses are not fully operationalized by the variables collected. 
A plan has been made to address this issue, and is covered later in this paper. 
