\subsection{Methodology}
The available data was filtered to group schools based on the median family income. 
For the year 2021, the median family income for the Charlotte region, according to the US census bureau, is \$62,817. 
As a loose grouping of schools, the schools where the median income was more than \$62,817 were classified as high socioeconomic groups and the schools where the median income was less than \$62,817 were grouped as low socioeconomic groups. 
The data was further filtered  creating datasets for before and after 2017 for each socioeconomic group. 

\subsection{Hypothesis 1}

In order to test hypothesis 1 which asserts that increasing the  numbers of mentors will eventually lead to higher college enrollment for low socioeconomic students, intention for enrolling in community college, intention for enrolling in senior institutions, AP participation rate, AP passing rate, and enrollment of disadvantaged students were considered for analysis and building model. 
Enrollment of disadvantaged students is considered as a response variable and the other variables are considered predictors. 
The correlation between the variables for both socioeconomic groups before and after  2017 were calculated. 
The correlation matrix showed that students from higher socioeconomic status, both AP participation and AP passing rate have negative correlation with intentions of enrolling in community college whereas the same variables 
have positive correlation with the intention of enrolling in senior public institutions. This indicates that the students from higher socioeconomic backgrounds who participated in AP courses are more likely to enroll in senior public institutions. 
For low socioeconomic student group, the enrollment variable and AP participation rate and AP passing rate have a higher correlation coefficient  for data after 2017. 
However, higher socioeconomic status students, AP and community college enrollment intentions are negatively or insignificantly positively correlated indicating that students from higher socioeconomic groups are less likely to intend to go to community college. 
For this group the correlation of AP with intention to enroll in senior public institutions is higher indicating that this group of students are more likely to intend to enroll in senior public institutions. 
A linear regression model was constructed using enrollment of disadvantaged as dependent variable and the rest of the variable as predictors. 
The model was tested for any multicollinearity among the predictor variables, the VIF test showed no indication of presence of multicollinearity. 
However, the parameter estimates associated with AP participation rate, AP passing rate, and community college intentions were statistically insignificant. 
Also log transformation on dependent variable indicated better linearity. 
So a final model was created, dropping the insignificant predictors and performing log transformation on the dependent variable. The model was statistically significant with improved adjusted R-square and lower standard error. 
The model passed normality and constant variance assumptions.

Correlation coefficients between the predictors and the response variable in the data before and after 2017 is shown in table \ref{tab:correlationtable}.
\\

\begin{threeparttable}
    \caption{Correlation Coefficients} %% the caption, required by APA7 and it looks nice :D
\label{tab:correlationtable} %% use this to refer to the table within the paper (I think a link will open up)
    \begin{tabular}{>{\centering\arraybackslash} p{0.16\linewidth} p{0.2\linewidth} p{0.2\linewidth} p{0.16\linewidth} p{0.16\linewidth}}      
    \toprule %% A line across the top
    Dependent                        & \multicolumn{4}{c}{Independent} \\ %% The titles of the columns.
                        \cmidrule(r){2-5} %% a line going across columns 2 - end
                            & Intention of Community College & Intention of Public Senior Institutions & AP Participation Rate & AP Passing Rate \\ %% The titles of the columns. 
                            
\midrule
College                     &        &  \textbf{Pre 2017}   &         &         \\     
Enrollment of               &  \large0.222& \large-0.048        &  \large-0.18  & \large-0.609  \\
Economically& && &\\      \cmidrule(r){2-5}
Disadvantaged               &        &  \textbf{Post 2017}  &         &  \\
Students                    & \large0.014  &\large0.081 & \large0.1182 & \large0.112  \\
\midrule
\end{tabular}

\end{threeparttable}

The correlation coefficients among the dependent and independent variables before 2017 are all negative indicating the predictors were not positively influencing the enrollment of disadvantaged students. 
However, the correlation coefficient among the same variables after 2017 are all positive indicating that these predictors are starting to positively influence the response variable, enrollment of disadvantaged students.

The model for data after 2017 also shows no presence of multicollinearity. 
However, the summary of the model shows that none of the predictors are  significant in explaining the variability in the response variable. 
This is an issue our team is going to look into at depth in the next project. 

\subsection{Hypothesis 2}

Hypothesis 2 currently has no data on CTE enrollment and credentials earned before 2017. K-Nearest Neighbors (KNN) was discussed as a method for filling in this missing data, however, it is getting trained on only data post-2017. 
This is after the creation of LOO, which would be training data before the recommendations on data made after the recommendations. A KNN model was decided against due to this reason, as the current objective is to compare pre-2017 with post-2017 data. 
This missing data presents a problem for testing hypothesis 2. Plans for dealing with this issue are addressed later in this paper. 


\subsection{Theoretical Analysis}

% Link your findings from the analysis back to your theoretical expectations at a conceptual level, as described in your hypothesis graph. 
While constructing the hypothesis model at conceptual level, we expected that increasing the number of mentors or guidance counselors would eventually lead to higher enrollment of economically disadvantaged students. 
As the literature reviews suggested that students make better decisions when they are given the right information, and students from low socioeconomic status often lack right information for making right decision for their education choice,  we expected that increasing the number of  mentors would lead to a higher number of economically disadvantaged students choosing AP courses which would lead to higher enrollment of these students. The correlation between AP stats and enrollment of disadvantaged students was negative for data before 2017, whereas the correlation is positive for data after 2017. However, the magnitude of the relationship is small. We still  consider this as a positive sign which has the potential to make a difference in the lives of students from low socioeconomic status in the long run. For the second hypothesis our expectations have not been tested yet due to the problem of data availability. However, our group is making serious efforts to obtain relevant data from CMS and conduct the analysis. The unexpected situation created by COVID pandemic might pose a serious challenge to our effort in securing reliable data. We expect to obtain the CTE data from CMS and conduct exploration on the data to find out whether the CTE enrollment and certification has improved among the economically disadvantaged students. 

% Link Findings to theoretical expectations
%  What are the implications of your findings for how readers should think about the economic mobility problem?  
Economic mobility is influenced by several factors including early care and education, college and career readiness, child and family stability, segregation based on race, gender and wealth, and social capital resources \parencite[][p. 3]{LOO}. 
Our group is focusing on a  few of the recommendations from Charlotte-Mecklenburg opportunity task force's recommendations on college and career readiness. As a result, our study is not going to shed light on the entirety of economic mobility. 
It is important for readers to understand that our study does not carry the overall situation of economic mobility for low socioeconomic students. 
However, this can encourage future researches on all fronts to actually understand whether the recommendations on each factor associated with upward mobility have been taken into serious consideration or not by the responsible parties in Charlotte-Mecklenburg region.

% Return back to the challenge for this project – did you address the challenge of understanding what has happened since the Task Force Report? 

The current analysis has only brushed the surface of the disparities that exist economically between students and high schools in CMS. 
Some evidence was found to indicate that the recommendations are having an impact, but why, where, and how have not been addressed. 
The team's hypotheses, feature selection, and analysis are relevant to reaching the goal of understanding how Charlotte has changed since the formation of Leading On Opportunity to oversee the implementation of the Task Force's recommendations. 
However, the task is not complete without more concepts and features related to Career and College Readiness, and in turn, upward mobility. 
Plans on further hypothesis development and data collection are addressed later in this paper.


% What problems or issues arose in the study that you think are important to consider?
Throughout our course of project, availability of data has been a very difficult problem. Publicly available data does not contain data beyond teachers and student demographics. This barely helps in exploring the impacts of career readiness factors influencing economic mobility mentioned by LOO report. 
Also, at the current level of data, AP stats have been shown  to have no significant influence on the enrollment of economically disadvantaged students. 
We have used AP stats as the predictors of enrollment of economically disadvantaged students. 
This situation has led us to rethink whether AP is a good predictor of enrollment for low socioeconomic students. 