There are a few potential analysis methods that could be implemented to help validate our hypotheses. The Chi-Squared Test can be used to help test the relationship between predictor and target variables. The evaluation metric, p-value based on degrees of freedom, can provide a statistically significant (or not significant) answer whether a predictor variable is related to the targets. 
Regression analysis is perfect for continuous variables, of which the majority of the variables being studied are. The few discrete variables that exist (such as number of counselors) can be divided into the student population, or another population statistic, to create a continuous variable. 
Regression allows for evaluation metrics such as the root mean squared error, correlation coefficients, and mean absolute error. 
It can be implemented with machine learning models or multivariate equations. 
Random Forest models are generally strong and simple to put into practice and can perform regression. Correlation can be helpful in determining the direction of a relationship between variables. 
The team is still looking into final analysis techniques. 
Regression and the Chi-Squared are some of the methods that seem to fit with the type of data and problem we are approaching. 
More machine learning models are being reviewed for fit with the research question and variables that operationalize it.
Regression analysis is a definite tool to utilize, but as to how it will be implemented remains a question. 
The hypotheses suggest a relationship between variables. 
Correlation coefficients can be an important tool in determining the validity of a hypothesis.