America has long been hailed as the land of opportunity, but is the age-old saying relevant equally across all parts of the US today?  
A study in 2014 by Raj Chetty and his colleagues, leveraging the possibilities of big-data, showed that opportunities for upward mobility (moving from lower fifth quartile of income to upper fifth income group) is unevenly distributed across different commuting zones in the US \parencite[][p. 1]{chetty2014}. 
Some major cities provide higher chances of upward mobility while some cities have a very low rate of upward mobility. The study showed that Charlotte, a booming economic hub of the southeast, has the least opportunity for upward mobility among fifty major cities in the US \parencite[][p. 1]{chetty2014}. 
The revelation took the city leaders by surprise. They scrambled to address the situation and  formed a task force consisting of experts from different walks of life. 
After more than a year of intense study and consultations, the task force  identified Early Care and Education, College and Career Readiness, and Child and Family Stability as the key determinants that influenced the children's ability to move up the economic ladder later in their lives. 
The task force also pointed out that the impacts of Segregation, and Social Capital cut across all three determinants mentioned above. 
With a goal to minimize the impacts of these determinants, the task force made several recommendations and implementation tactics. This paper is focusing on  whether the recommendations made by the Task Force to address the issue of College and Career Readiness are being implemented so far. 
If they have been implemented, what have their impacts been?
The task force report has placed a huge emphasis on the role of guidance counselors as most of the students from low socioeconomic backgrounds lack not just resources but also right information on the cost of attending college, admission procedure, and the quality of education institution. 
This situation can often lead the students to avoid college or apply for college with poor records in educational quality and degree completion \parencite[][]{castleman2014intensive}.
Increasing the number of counselors and providing families and students from low socioeconomic backgrounds with necessary information and resources can significantly help them in making right decisions about their academic journey \parencite[][]{castleman2014intensive}.
