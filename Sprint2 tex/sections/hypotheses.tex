\textbf{Hypothesis 1} \textit{As the number and quality of mentors increase, there will be an increase in academic performance and college acceptance rates for low-socioeconomic students.}

Active intervention by counselors have shown to have a significant positive impact on mathematics test scores and attitudes towards school \parencite[][]{lee1993}. 
Guidance counselors assist in the college application and enrollment process \parencite[][]{deslonde2018high,tang2019high}.
For low-socioeconomic schools, the counselor's duties expand to help students navigate any roadblock that may arise to hinder postsecondary education enrollment \parencite{farmer2006, deslonde2018high}. 
The experience and pool of knowledge well-trained counselors have can give students access to social capital than they would otherwise not be able to utilize \parencite{tang2019high}.

The variables that will operationalize this hypothesis are the number of guidance counselors in Charlotte-Mecklenburg Schools, as well as high school graduate intention. 
To operationalize academic performance, the percentage of the student population enrolled in Advanced Placement classes, and the percentage of exams that score at least a 3 out of 4 will be used. 
College enrollment rates are fairly straightforward, and economically disadvantaged students enrolling in college will be used. Dual enrollment is a sought after variable to help measure academic performance. 
\\


\textbf{Hypothesis 2} \textit{As the quality and number of mentors increase, there will be an increase in obtained CTE Credentials by students within Charlotte Mecklenburg County Schools.}

The literature suggests that deliberate intervention with academic progression amongst students, has a direct correlation with academic performance, career pathway selection, and career progression. 
For example, students who have involvement within their educational career starting in early childhood, do indeed have a higher likelihood of climbing the ladder of economic mobility \parencite[][]{magnuson2000}.
Thus, educational involvement and academic guidance at an early age, including transitions into high school enable students to perform well academically and be conscientious of their career selection, including careers in technical education. 
Consequently, the consideration and pursuit of numerous career options engenders career awareness for students who may prefer vocational training over that of postsecondary education. 
It is suggested that, causally, early childhood habits help identify vocational preferences, competence, and parameters of success \parencite[][]{magnuson2000}.
The catalyst for economic prosperity and upward mobility is not strictly causal to postsecondary education efforts. 
Vocational training and Trades are very positive markers and are commonly overlooked as a viable option. 
The literature suggests that parents, educators, and counselors should emphasize the pursuit of vocations within the United States, primarily because of its dichotomy of historical attrition and increased necessity; this ideal was additionally supported by the Perkins Career and Technical Education (CTE) Act of 2006 \parencite[][]{castellano2017}.
The variables that will be operationalized to measure this hypothesis will come from the \textit{NC Department of Public Instruction Data} that provides insight into obtained CTE credentials of students within Charlotte Mecklenburg County Schools, by year (CTE\_Credentials, CTE\_Enrollment); and also the aggregate and ratio of counselors and mentors within Charlotte Mecklenburg County Schools (Staff).
