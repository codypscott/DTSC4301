The strategies and implementation tactics proposed by Opportunity Task Force appear in line with the scholarly literature. 
From increasing the number of guidance counselors to provide insight into academic and career options, to increasing the accessibility of AP courses to students from low socioeconomic backgrounds, the literature suggests that those mitigation measures in fact have shown positive effects on students \parencite[][]{castleman2014intensive}.
It is expected to see a positive correlation between the number of guidance counselors and college acceptance rates. 
Similarly, a positive relationship between counselors and the number of students obtaining CTE credentials is an expectation. 
With the implementation of a linear regression model to establish the relationship between these variables, it is anticipated to see a positive value for coefficient of the dependent variable. 
Other machine learning models are still under consideration, but the team expects to find results consistent with the hypothesis that have been presented in this paper. 
However, the details of those explorations are still underway and the team is actively looking for data that could continue to help establish those relationships. 
The next step for our team will be deciding the appropriate analysis methods and machine learning models for future prediction. 
Further research into the domain via scholarly resources is necessary to support future methods, models, and evaluation metrics that will be implemented.
The next sprint will include details of the results from the analysis and models that we will be deciding to use for making predictions, so these decisions are the topic of discussion amongst the team.