\subsection{Data Sources and Sets}
The North Carolina Department of Public Instruction (NCDPI) is tasked with carrying out education legislation (NC DPI). It oversees public schools, charter schools, and schools for students with hearing and vision impairments. 
It not only develops the curriculum taught in North Carolina, but also provides leadership and support in accountability, finance, and administrative work \parencite[][]{NCDPI}.
The NCDPI also provides licenses to teachers in the state \parencite[][]{NCDPI}. It coordinates and collaborates with school administration and accountability offices to collect and organize data on schools state-wide \parencite[][]{NCDPI}.

NCDPI utilizes school reporting data to compile the annual School Report Card. The Report Cards go over topics related to primary and high school performance \parencite[][]{NCReportCards}.
Readily available for download is information ranging from 2013-2020, all in Microsoft Excel format. Various information is covered by county, including charter and regional schools \parencite[][]{NCReportCards}. 
The NC Report Cards contain information on academic performance, such as Advanced Placement (AP) classes and exams, college enrollment, and Career and Technical Education (CTE) programs\parencite[][]{NCReportCards}.

The North Carolina Public Schools Statistical Profile is intended to supply statistical information on the public school system, but also contains information on charter and regional schools\parencite[][]{NCStats}. 
It has a variety of information at the state, district, and school level\parencite[][]{NCStats}.
It provides “general statistical data to the public, professional educators, and the General Assembly” \parencite[][]{NCStats}. 
For school personnel, the number of individuals in each position are given across the school district. High school graduates are broken down by demographic and post-high school intentions, whether that is college, employment, or something else\parencite[][]{NCStats}.

\blockquote{The Institute of Social Capital (ISC) was founded in 2004 by stakeholders in the Charlotte-Mecklenburg region. It became part of the UNC Charlotte Urban Institute in 2012. 
The ISC provides unidentified administrative data and assistance in the research and data analysis efforts. To access the data, one must provide a request. 
The intention is to use this information to gain the percentage of first-generation college students dually-enrolled at CPCC during their junior and senior high school years. 
Also, to gain the information on the percentage of CMS students in paid internships, if possible. This information may be elsewhere, or not available. (\cite{team1a})}

The Quality of Life Explorer takes a closer look at societal, structural, and economic conditions within Mecklenburg County \parencite[][]{QualityofLife}.
It provides information in the form of maps, tables, and reports, broken down by neighborhood. Reports can be created by filtering the data geographically, allowing for custom geospatial analysis \parencite[][]{QualityofLife}.
Specifically, the Quality of Life Explorer has median household income by zip code. This allows the team to find the median household income where each school is located, giving some glimpse at the economic level of each institution \parencite[][]{QualityofLife}.

\subsection{Missing}
This research focuses on high school alone out of primary and secondary education. Beginning early in life planning can have a great impact on the success of a student \parencite[][]{magnuson2000}. 
However, the decisions finalized in high school can have lasting more immediate effects on a student. 
The Chetty report was published in 2014 and the Leading on Opportunity Report was published in early 2017 \parencite[][]{chetty2014, LOO}. Those entering high school or graduating since 2017 are the closest to making those lifelong decisions. 
If effective change is to be made in the last few years to positively alter one of these students' choices, focusing on the success of high school is most important. 
The data necessary to perform accurate analysis on the academic performance and job readiness of students needs to cover all high schools in Mecklenburg County. 
Academic measures, such as test scores and college enrollment per school can help measure the college readiness of the graduating students. The quality and number of mentors to provide guidance and access to social capital for students is integral to increasing the education attainment of Charlotte-Mecklenburg students. 
Currently, cohesive and comprehensive data on measures related to career readiness, such as if students are given access to paid internships or how many of a graduating class seeking employment find it, is lacking. 
Information on high school students taking college courses is outdated (2017 and before), and no longer kept by the sources we have investigated. 
This is a loss to understanding the effectiveness of the `Career and College Promise Program' which allows students to earn college credits at CPCC during high school. 
Without more solid information on career-related data, measuring how well-equipped students are to thrive in a professional environment is mostly about how many skills they are taught while still within reach of the Charlotte-Mecklenburg School system. 

\subsection{Codebook}
Much of the data collected in the various data sets are reported by the school, due to accountability laws \parencite[][]{NCDPI}. 
If the data request from the ISC is successful, that information comes through collaborative collection efforts and administrative data. Refer to table\ref{tab:codebook} for the full codebook. As one can see, the years for some of our variables are lacking, such as information regarding CTE enrollment and credentials.

\subsection{Variable Information}
All the variables are numeric, allowing for the mean, standard deviation, minimum, maximum, non-missing, and percent of missing values to be calculated. Table \ref{tab:summarystats1} \& \ref{tab:summarystats2}  provide information on the statistics of the variables. Identifier variables are covered in Table \ref{tab:idvariables}. These can be found in the appendix.

Several or more years of data are missing for some key variables, such as CTE credentials earned and students enrolled in a CTE program. 
College enrollment is not complete for every subgroup and every year. Most importantly, the 2020 school year enrollment data is missing. 
It also appears that the 2016 year is missing for this variable, as well. Without more comprehensive CTE data, potential analysis on the measures that increase the probability of success for students entering the workforce is potentially crippled. 
Going forward, substitutes may need to be located, or innovative and accurate methods of completing missing data should be used. 